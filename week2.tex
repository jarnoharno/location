\documentclass[a4paper,parskip=full]{scrartcl}
\usepackage[finnish]{babel}
\usepackage[utf8]{inputenc}
\usepackage[T1]{fontenc}
\usepackage{mathptmx}
\usepackage{hyperref}
\usepackage{enumitem}
\usepackage{calc}
\usepackage{amsmath}
\usepackage{siunitx}
\usepackage{tabularx}
\usepackage[numbers]{natbib}
\usepackage{hyperref}
\usepackage{verbatim}
\setlist[description]{font={\rmfamily}}
\addtokomafont{disposition}{\rmfamily}
\sisetup{per-mode=symbol}

\subject{Location-Awareness: Spring 2014}
\author{Jarno Leppänen}
\title{Exercise: 2}
\date{25.3.2014}

\begin{document}

\maketitle

1) \textbf{Positioning}

a)

\begin{tabularx}{\textwidth}{ | >{\raggedright\arraybackslash}p{14ex} | >{\raggedright\arraybackslash}X | >{\raggedright\arraybackslash}X | }
  \hline
  & \textbf{Ultrasound} & \textbf{GSM} \\ \hline
  \textbf{Location representation} &
    Relative if positions of receivers/transmitters are not fixed (eg.
    Relate). Absolute if transmitters/receivers are fixed and their positions in
    some coordinate system are known (eg. Cricket with known beacon positions).
    &
    Absolute \\ \hline
  \textbf{Scale} & Local, typically indoors & Local, areas with required
    infrastructure, outdoors \\ \hline
  \textbf{Location system type} & Can be of any type, depending on the system
    & Can be of any type, depending on client/infrastructure support \\ \hline
  \textbf{Measurement} & Ultrasound signal, measurement type can be
    ID, RSS or time of flight, depending on the system. & Radio signal,
      base station IDs and RSS-measurements are typically used in
      client-based systems, time of flight (timing advance) can be used with
      network-based systems. \\ \hline
  \textbf{Error sources} & Noise, geometry, clock errors, multipath propagation
    & Noise, geometry, clock errors, multipath propagation, signal
      attenuation \\ \hline
\end{tabularx}

b) The distance traveled is $\SI{3}{m/s} \cdot \SI{18}{h} \cdot \SI{3600}{s/h}
= \SI{194400}{m}$. Unix program called geod can be used to calculate the
final position and azimuth.
\begin{verbatim}
$ echo "39.0291382dN 125.7421118dE 145.6d $(echo '3 * 18 * 3600' | bc)" |\
> geod +ellps=WGS84 +units=m
37d34'53.553"N  126d59'13.128"E -33d37'41.828"
\end{verbatim}
The new location is N\ang{37;34;53.553} E\ang{126;59;13.128}.

2) \textbf{RF-based positioning}

a) Quite frankly, the figure in this assignment sucks. I'm assuming that (i) the
boat and the shore in the figure have changed places, (ii) the line between them
is orthogonal to the line between the two reference points and that (iii)
the shore is lying at the intersection of the aforementioned orthogonal lines.

The first assumption does not change the outcome of the calculations
but makes the setting a bit more realistic. Why would the boat lie exactly
between the two reference points? Why would we know the distances
between shore and the two reference points, but not between the boat and
any other place?

With the assumptions We get $\sin\alpha = \frac{d}{a}$ so the the distance to
the shore is $d = a\sin\alpha$. Plugging in values we get $d = \SI{100}{m}
\cdot \sin\ang{60} \approx \SI{86.6}{m}$.

Let the distances between shore
and reference points A and B e $x$ and $y$ respectively and the distance
between reference points $z = x + y$. Now $\cos\alpha = \frac{x}{a}$ and
$\cos\beta = \frac{y}{b}$. Solving $z$ we get $z = a\cos\alpha + b\cos\beta$.
Plugging in values we get the distance $z = \SI{100}{m} \cdot \cos\ang{60} +
\SI{150}{m} \cdot \cos\ang{40} \approx \SI{165}{m}$.

b) R was used for the calculations.
\begin{verbatim}
> e=c(14.42,13.77,17.55,12.01,17.34,13.64,51.93,51.72,61.51,34.14)
> quantile(e, c(.5,.95))
   50%    95%
   17.445 57.199
\end{verbatim}
The median error is 17.445 and 95-percentile error is 57.199.

c) With R again:
\begin{verbatim}
> e=c(16.62,12.87,18.35,11.42,17.62,13.04,52.13,50.62,63.01,32.94)
> quantile(e, c(.5,.95))
   50%    95%
   17.985 58.114
\end{verbatim}
The median error is 17.985 and 95-percentile error is 58.114.

Assuming that the units in these measurements are comparable (meters?),
the error distributions are nearly identical. But because the first system
seems to be just a bit more accurate, it is better for navigation. If some
inaccuracy is expected from a treasure hunt game, the second might be better
for that purpose.

3) \textbf{Fingerprint-based positioning}

a) Positions were calculated with R. The program can be found at
\url{http://www.cs.helsinki.fi/u/jaleppan/location}.
With all invocations $k = 5$.
\begin{verbatim}
$ ./knnpos.R 5 -70 -72
32.33644 184.2327
\end{verbatim}
The estimated position with kNN is approximately 32.3, 184.2.

b)
\begin{verbatim}
$ ./knnpos.R 5 -70 -72 TRUE
36.19177 160.3137
\end{verbatim}
The estimated position with WkNN is approximately 36.2, 160.3.

4) \textbf{Dilution of Precision}
a, b) DoP matrix and values were calculated with R. The program can be found at
\url{http://www.cs.helsinki.fi/u/jaleppan/location}.
\begin{verbatim}
$ ./dop.R
 1.6744270 -0.2895913 -1.7609931 -1.5059021
-0.2895913  0.9235268  0.8591840  0.6345951
-1.7609931  0.8591840  9.1252271  6.3591522
-1.5059021  0.6345951  6.3591522  4.7398650
HDOP 1.611817
VDOP 3.020799
PDOP 3.423913
\end{verbatim}

5) \textbf{Positioning implementation}

I implemented a probabilistic fingerprinting algorithm based on normal
distribution with R. Missing RSS values were replaced with an RSS value of $-120$.
The program can be found at \url{http://www.cs.helsinki.fi/u/jaleppan/location}.
\begin{verbatim}
$ ./prpos.R test_set.csv 
29 30 47 1 46 36 47 17 46 18
\end{verbatim}
The 10 test measurements are estimated to be at grid cells 29, 30, 47, 1, 46,
36, 47, 17, 36 and 18 in this order.

%\bibliographystyle{plainnat}
%\bibliography{ref}

\end{document}
