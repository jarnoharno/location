\documentclass[a4paper]{scrartcl}
\usepackage[finnish]{babel}
\usepackage[utf8]{inputenc}
\usepackage[T1]{fontenc}
\usepackage{mathptmx}
\usepackage{hyperref}
\usepackage{enumitem}
\usepackage{calc}
\usepackage{amsmath}
\usepackage{siunitx}
\setlist[description]{font={\rmfamily}}
\addtokomafont{disposition}{\rmfamily}

\subject{Location-Awareness: Spring 2014}
\author{Jarno Leppänen}
\title{Exercise: 1}
\date{18.3.2014}

\begin{document}

\maketitle

\begin{enumerate}
  \item
    \begin{enumerate}
      \item
        \begin{description}[style=nextline]
          \item[IndoorAtlas]{
              IndoorAtlas is an indoor mapping and positioning service based in
              Oulu, Finland. Their rather unique positioning technology is based
              on mapping magnetic fingerprints inside buildings.
            }
          \item[iBeacon]{
              iBeacon is a Bluetooth Low Energy (BLE) based specification from
              Apple which can be used for indoor positioning and proximity
              sensing. BLE beacons have very low power consumption and
              can operate up to 2 years with a single charge.
            }
          \item[ByteLight]{
              ByteLight is a location tracking system based on LED light pulses
              detected with a smartphone app.
            }
          \item[WiFiSLAM]{
              WiFiSLAM was a startup that developed sensor fusion technology
              which combined radio fingerprinting with inertial navigation
              done with smartphone sensors. They were acquired by Apple in 2013.
            }
        \end{description}
      \item
        \begin{description}[style=nextline]
          \item[Tinder]{
              Tinder is a location based social networking app that connects
              mutually interested users. Tinder analyzes users' social graph
              based on their Facebook profiles and matches candidates based
              on current location, mutual friends and common interests.
            }
        \end{description}
    \end{enumerate}

  \item
    \begin{enumerate}
      \item
        A geodetic datum is a reference from which measurements are made.
        Horizontal datums are used to describe a point on the Earth's surface
        in terms of latitude and longitude. Vertical datums are used to measure
        depths and elevations.
      \item
        Planar or azimuthal map projections project the Earth on a plane. They
        share the property that directions from the projections central point
        are preserved.

        Conic projections map meridians to equally spaced lines radiating from
        the apex of the projection and parallels to circular lines centered on
        the apex. Conic projections are accurate around the two parallels where
        the surface of the Earth and the cone intersect.
      \item Inverse flattening $f^{-1}$ is defined in terms of semimajor axis
        $a$ and semiminor axis $b$ as

        \begin{align}
          f^{-1} &= \frac{a}{a - b}
        \end{align}

        Solving $a$ and plugging in given values
        $b = 6356752.31424518\ \textup{m}$ and
        $f^{-1} = 298.257223563$ we get

        \begin{align}
          b &= \frac{b f^{-1}}{f^{-1} - 1} \\
            &\approx 6378137\ \textup{m}
        \end{align}


\end{enumerate}


\end{document}
